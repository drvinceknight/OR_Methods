\documentclass[12pt]{article}

%packages
\usepackage{graphicx}
\usepackage{amsmath}
\usepackage{mathdots}
\usepackage{amsthm}
\usepackage{amssymb}
\usepackage{fancyhdr}
\usepackage{pstricks}
\usepackage{pst-node}
\pagenumbering{arabic}
\usepackage{hyperref}
\usepackage{lscape}
%Margins etc...
\setlength{\textheight}{240mm}
\setlength{\topmargin}{-17mm} \setlength{\oddsidemargin}{-4mm}
\setlength{\textwidth}{166mm} \setlength{\parindent}{0mm}
\setlength{\marginparsep}{9mm} \setlength{\parskip}{3mm}

\begin{document}
\begin{center}
\Huge{Markov Chains Sage Sheet Solutions}\\
\tiny{Last updated: \today.}
\end{center}

\begin{enumerate}
\item Discrete Markov Chains\\
Navigate to the Sage code snippet at: \url{http://interact.sagemath.org/node/39} and perhaps start a new Sage worksheet so that you can try a few of your own Sage commands.

WARNING: The convention used in this code snippet is different to the notes. The columns sums add to 1 not the row sums. Keep this in mind as you work through it.
ANSWER:\\(The solutions here make use of the convention in the notes)

\begin{enumerate}
\item Using the default inputs, what is the steady state distribution associated with this Markov chain (try and use the Sage ``solve" command to verify this)?

ANSWER:\\
The steady state distribution is given by:

$$\pi=\left({75\over 2},{75\over 2},25\right)$$

We can obtain this using the following Sage code:

\begin{verbatim}
var("pi1,pi2,pi3")
solve([.8*pi1+.2*pi2==pi1,.1*pi1+.7*pi2+.3*pi3==pi2,
       .1*pi1+.1*pi2+.7*pi3==pi3,pi1+pi2+pi3==50+30+20],[pi1,pi2,pi3])
\end{verbatim}

we can make things a bit more ``generic'' as follows:
\begin{verbatim}
A=matrix([[.8,.1,.1],[.2,.7,.1],[0,.3,.7]])
var("pi1,pi2,pi3")
solve([A[0,0]*pi1+A[1,0]*pi2+A[2,0]*pi3==pi1,
       A[0,1]*pi1+A[1,1]*pi2+A[2,1]*pi3==pi2,
       A[0,2]*pi1+A[1,2]*pi2+A[2,2]*pi3==pi3,
       pi1+pi2+pi3==50+30+20],[pi1,pi2,pi3])
\end{verbatim}

\item How long does it seem to take to arrive at that state (try and use Sage to verify this)?

ANSWER:\\
It seems to take about 20 time steps. We can check this by trying the following code with various values of $n$:

\begin{verbatim}
n=11
A=matrix([[.8,.1,.1],[.2,.7,.1],[0,.3,.7]])
pi_0=vector([50,30,20])
pi_0*A^n
\end{verbatim}

\end{enumerate}
\item Continuous Markov Chains\\
The evolution of a Continuous Markov Chain obeys the following differential equation:

$${d\pi(t)\over dt}=\pi(t)Q$$
which has solution:
$$\pi(t)=\pi(0)e^{Qt}$$
This requires the calculation of the exponential of a Matrix. This is not straightforward! One approach is to use the following formula:
$$
e^{M}=\mathbb{I}+\sum_{k=1}^{\infty}{M^k\over k!}
$$
Navigate to the Sage code snippet at \url{http://interact.sagemath.org/node/71} and experiment with the code to see how good this approach is.

ANSWER:\\
We see that the default inputs require 21 terms to give a good approximation (to 4 decimal places). Playing around with different inputs should give different results.

\begin{enumerate}
\item Consider the following Markov Chain:
$$Q=\begin{pmatrix}
-3&3&0&0\\
4&-7&3&0\\
0&4&-7&3\\
0&0&4&-4
\end{pmatrix}$$
\item Using the Sage ``solve" command to obtain the steady state probabilities for this Markov Chain.

ANSWER:\\
The steady state probabilities are given by:

$$\pi=(.3657,.2743,.2057,.1543)$$

This can be obtained using the following code:

\begin{verbatim}
Q=matrix([[-3,3,0,0],[4,-7,3,0],[0,4,-7,3],[0,0,4,-4]])
var("pi1,pi2,pi3,pi4")
solve([Q[0,0]*pi1+Q[1,0]*pi2+Q[2,0]*pi3+Q[3,0]*pi4==0,
       Q[0,1]*pi1+Q[1,1]*pi2+Q[2,1]*pi3+Q[3,1]*pi4==0,
       Q[0,2]*pi1+Q[1,2]*pi2+Q[2,2]*pi3+Q[3,2]*pi4==0,
       Q[0,3]*pi1+Q[1,3]*pi2+Q[2,3]*pi3+Q[3,3]*pi4==0,
       pi1+pi2+pi3+pi4==1],[pi1,pi2,pi3,pi4])
\end{verbatim}
\end{enumerate}
\end{enumerate}

\end{document}
