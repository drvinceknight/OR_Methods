\documentclass[12pt]{article}

%packages
\usepackage{graphicx}
\usepackage{amsmath}
\usepackage{mathdots}
\usepackage{amsthm}
\usepackage{amssymb}
\usepackage[dvips]{color}
\usepackage{fancyhdr}
\usepackage{pstricks}
\usepackage{pst-node}
\pagenumbering{arabic}
\usepackage{hyperref}
\usepackage{lscape}
%Margins etc...
\setlength{\textheight}{240mm}
\setlength{\topmargin}{-17mm} \setlength{\oddsidemargin}{-4mm}
\setlength{\textwidth}{166mm} \setlength{\parindent}{0mm}
\setlength{\marginparsep}{9mm} \setlength{\parskip}{3mm}

\begin{document}
\begin{center}
\Huge{Game Theory Exercise Sheet SOLUTIONS}\\
\tiny{This sheet was last updated on \today.}
\end{center}

\begin{enumerate}
\item

$$
\begin{array}{@{}c@{\hspace{2cm}}c@{}}
\begin{tabular}{|c|c|c|c|}
\hline
 &$s_1$      &$s_2$  &$s_3$   \\\hline
$r_1$&$(\underline{6},\underline{3})$&$(2,2)$&$(2,2)$\\\hline
$r_2$&$(4,0)$&$(0,3)$&$(\underline{4},\underline{5})$\\\hline
$r_3$&$(2,3)$&$(\underline{3},\underline{4})$&$(3,2)$\\\hline
\end{tabular}&
\begin{tabular}{|c|c|c|}
\hline
 &$s_1$      &$s_2$    \\\hline
$r_1$&$(\underline{7},-2)$&$(\underline{4},\underline{0})$\\\hline
$r_2$&$(1,-5)$&$(0,\underline{-4})$\\\hline
$r_3$&$(4,\underline{-1})$&$(3,-5)$\\\hline
$r_4$&$(6,-7)$&$(\underline{4},\underline{-5})$\\\hline
\end{tabular}\\[2mm]
(a)&(b)\\[2mm]
\begin{tabular}{|c|c|c|c|}
\hline
 &$s_1$      &$s_2$  &$s_3$  \\\hline
$r_1$&$(160,\underline{2})$&$(\underline{205},\underline{2})$&$(44,\underline{2})$\\\hline
$r_2$&$(175,1)$&$(180,.5)$&$(45,\underline{5})$\\\hline
$r_3$&$(\underline{201},3)$&$(204,4)$&$(\underline{50},\underline{10})$\\\hline
$r_4$&$(120,4)$&$(107,\underline{6})$&$(49,2)$\\\hline
\end{tabular}&
\begin{tabular}{|c|c|c|c|}
\hline
 &$s_1$      &$s_2$  &$s_3$  \\\hline
$r_1$&$(0,0)$&$(-1,1)$&$(1,-1)$\\\hline
$r_2$&$(1,-1)$&$(0,0)$&$(-1,1)$\\\hline
$r_3$&$(-1,1)$&$(1,-1)$&$(0,0)$\\\hline
\end{tabular}\\[2mm]
(c)&(d)

\end{array}
$$

Since the number of Nash Equilibria for any given game is odd, we expect to not have identified all equilibria for $(b),(c)$ and $(d)$.

\item The bi-matrix representation is given by:
\begin{center}
\begin{tabular}{|c|c|c|c|c|c|c|}
\hline
     &$100$  &$99$    &$98$&\dots&3&2  \\\hline
$100$&$(100,100)$&$(97,101)$&$(96,100)$&$\dots$&$(1,5)$&$(0,4)$\\\hline
$99$&$(101,97)$&$(99,99)$&$(96,100)$&$\dots$&$(1,5)$&$(0,4)$\\\hline
$98$&$(100,96)$&$(100,96)$&$(98,98)$&$\dots$&$(1,5)$&$(0,4)$\\\hline
$\vdots$&$\dots$&$\dots$&$\dots$&$\ddots$&$\vdots$&$\vdots$\\\hline
$3$&$(5,1)$&$(5,1)$&$(5,1)$&$\dots$&$(3,3)$&$(0,4)$\\\hline
$2$&$(4,0)$&$(4,0)$&$(4,0)$&$\dots$&$(4,0)$&$(2,2)$\\\hline
\end{tabular}
\end{center}

This game is immediate to solve with dominance and so the Nash equilibrium is $(2,2)$.

\item We have the bi-matrix game representation:
\begin{center}
\begin{tabular}{|c|c|c|c|}
\hline
 &$R$      &$P$  &$S$  \\\hline
$R$&$(0,0)$&$(-1,1)$&$(1,-1)$\\\hline
$P$&$(1,-1)$&$(0,0)$&$(-1,1)$\\\hline
$S$&$(-1,1)$&$(1,-1)$&$(0,0)$\\\hline
\end{tabular}
\end{center}
There is no pure Nash equilibrium and it is immediate to see that no mixed strategy will have support of size 2. Indeed, assume that a mixed strategy for player 1 does not play ``scissors''. Player 2 would have an immediate benefit of playing the pure strategy ``paper'' (as he'll never lose). This can be shown mathematically.\\

Thus the mixed strategy for player 1, $\rho$, will be of the form:

$$\rho=(p,q,1-p-q)$$

The mixed strategy for player 2, $\sigma$, will be of the form:

$$\sigma=(u,v,1-u-v)$$

Using the equality of payoffs theorem, we have:

\begin{equation}u_1(R,\sigma)=u_1(S,\sigma)=u_1(T,\sigma)\label{RPS1}\end{equation}
and
\begin{equation}u_2(\rho,R)=u_2(\rho,S)=u_2(\rho,T)\label{RPS2}\end{equation}

We have:

\begin{equation}\begin{array}{@{}r@{\;}c@{\;}l@{\;}r@{}}
u_1(R,\sigma)&=&-v+1-u-v&(a)\\[2mm]
u_1(P,\sigma)&=&u-1+u+v&(b)\\[2mm]
u_1(S,\sigma)&=&-u+v&(c)\\[2mm]
\end{array}
\label{RPS2}
\end{equation}

Combining (\ref{RPS1}) and (\ref{RPS2}) gives:

$$\begin{array}{@{}r@{\;}c@{\;}l@{}}
(a)=(b)&\Rightarrow&3u+3v=2\\[2mm]
(a)=(c)&\Rightarrow&3v=1\\[2mm]
(b)=(c)&\Rightarrow&3u=1\\[2mm]
\end{array}$$

Thus $\sigma=\left({1\over3},{1\over3},{1\over3}\right)$ as expected. A similar approach using (\ref{RPS2}) gives the expected result for $\rho$.


\item
Recall:

\begin{center}
\begin{tabular}{|c|c|c|}
\hline
  &Attack Bomber 1&Attack Bomber 2    \\\hline
Transport with Bomber 1&$(80,-80)$     &$(100,-100)$\\\hline
Transport with Bomber 2&$(100,-100)$   &$(60,-60)$\\\hline
\end{tabular}
    \end{center}
There is clearly no pure Nash equilibria. Let the bombers use bomber 1 with probability $p$ (thus they use bomber 2 with probability $1-p$). We denote the mixed strategy of the bombers by $\rho=\{p,1-p\}$. Let the fighter attack bomber 1 with probability $q$ (thus the fighter attacks bomber 2 with probability $1-q$). We denote the mixed strategy of the fighter by $\sigma=\{q,1-q\}$. We could use the equality of payoffs theorem to solve this problem. Let us however, consider a direct approach by looking at best responses:
$$\begin{array}{@{}r@{\;}c@{\;}l@{}}
u_1(\rho,\sigma)&=&80pq+100(p(1-q)+q(1-p))+60(1-q)(1-p)\\[2mm]
                &=&20(3+2p+2q-3pq)\\[2mm]
                &=&20(p(2-3q)+3+2q)
\end{array}$$
We immediately see that:
\begin{itemize}
\item If $q<{2\over3}$ then player 1s best response is to choose $p=1$.
\item If $q>{2\over3}$ then player 1s best response is to choose $p=0$.
\item If $q={2\over3}$ then player 1s best response is to play any mixed strategy.
\end{itemize}
Similarly we have:
$$\begin{array}{@{}r@{\;}c@{\;}l@{}}
u_2(\rho,\sigma)&=&-(80pq+100(p(1-q)+q(1-p))+60(1-q)(1-p))\\[2mm]
                &=&-(20(3+2p+2q-3pq))\\[2mm]
                &=&20(q(3p-2)-3-2p)
\end{array}$$
and we have:
\begin{itemize}
\item If $p<{2\over3}$ then player 1s best response is to choose $q=0$.
\item If $p>{2\over3}$ then player 1s best response is to choose $q=1$.
\item If $p={2\over3}$ then player 1s best response is to play any mixed strategy.
\end{itemize}
The only strategies that are best responses to each other is $\rho=\sigma=\left({2\over3},{1\over3}\right)$.



\item Using the equality of payoffs theorem identify all the Nash equilibria for the following games:
(a)
\begin{center}
\begin{tabular}{|c|c|c|}
\hline
 &$s_1$      &$s_2$     \\\hline
$r_1$&$(0,0)$&$(2,1)$\\\hline
$r_2$&$(1,2)$&$(0,0)$\\\hline
\end{tabular}
\end{center}


The pure Nash equilibria are given by $(r_2,s_1)$ and $(r_1,s_2)$. Consider the mixed strategies $\rho=(p,1-p)$ and $\sigma=(q,1-q)$. By the equality of payoff theorem we have:
$$u_1(r_1,\sigma)=u_1(r_2,\sigma)$$
and
$$u_2(\rho,s_1)=u_2(\rho,s_2)$$

The first equation is equivalent to:
$$2(1-q)=q$$
which gives $q={2\over3}$. Similarly we get $p={2\over3}$. Thus $\rho=\sigma=\left({2\over3},{1\over3}\right)$.

(b)

\begin{center}
\begin{tabular}{|c|c|c|}
\hline
 &$s_1$      &$s_2$     \\\hline
$r_1$&$(3,3)$&$(3,2)$\\\hline
$r_2$&$(2,2)$&$(5,6)$\\\hline
$r_3$&$(0,3)$&$(6,1)$\\\hline
\end{tabular}
\end{center}

The pure Nash equilibria is $(r_1,s_1)$. Consider the mixed strategies $\rho=(p,q,1-p-q)$ and $\sigma=(u,1-u)$. The difficult part of this problem is to identify the various different supports that $\rho$ may have (it is obvious that the size of the support of $\sigma$ is 2). Let us first consider supports of size 2:

\begin{itemize}
\item Assume that the support of $\rho$ is $\{r_1,r_2\}$:\\
Using the equality of payoffs theorem we have:
$$u_1(r_1,\sigma)=u_1(r_2,\sigma)$$
and
$$u_2(\rho,s_1)=u_2(\rho,s_2)$$
this gives:
$$u_1(r_1,\sigma)=u_1(r_2,\sigma)\;\Rightarrow\;3(u+1-u)=2u+5(1-u)\;\Rightarrow\;u={2\over3}$$
and (recalling that in this case we have $\rho=(p,1-p,0)$)
$$u_2(\rho,s_1)=u_2(\rho,s_2)\;\Rightarrow\;3p+2(1-p)=2p+6(1-p)\;\Rightarrow\;p={4\over5}$$
Thus this support gives the mixed Nash equilibium: $\left(\left\{{4\over5},{1\over5},0\right\},\left\{{2\over3},{1\over3}\right\}\right)$

\item Assume that the support of $\rho$ is $\{r_2,r_3\}$:\\
Using the equality of payoffs theorem we have:
$$u_1(r_2,\sigma)=u_1(r_3,\sigma)$$
and
$$u_2(\rho,s_1)=u_2(\rho,s_2)$$
this gives:
$$u_1(r_2,\sigma)=u_1(r_3,\sigma)\;\Rightarrow\;2u+5(1-u)=0u+6(1-u)\;\Rightarrow\;u={1\over3}$$
and (recalling that in this case we have $\rho=(0,q,1-q)$)
$$u_2(\rho,s_1)=u_2(\rho,s_2)\;\Rightarrow\;3q+3(1-q)=6q+(1-q)\;\Rightarrow\;q={1\over3}$$
Thus this support gives the mixed Nash equilibium: $\left(\left\{0,{1\over3},{2\over3}\right\},\left\{{1\over3},{2\over3}\right\}\right)$

\item Assume that the support of $\rho$ is $\{r_1,r_3\}$:\\
Using the equality of payoffs theorem we have:
$$u_1(r_1,\sigma)=u_1(r_3,\sigma)$$
and
$$u_2(\rho,s_1)=u_2(\rho,s_2)$$
this gives:
$$u_1(r_1,\sigma)=u_1(r_3,\sigma)\;\Rightarrow\;3u+3(1-u)=0u+6(1-u)\;\Rightarrow\;u={1\over2}$$
and (recalling that in this case we have $\rho=(p,0,1-p)$)
$$u_2(\rho,s_1)=u_2(\rho,s_2)\;\Rightarrow\;3p+3(1-p)=2p+(1-p)\;\Rightarrow\;p=2$$
However, this last value is not consistent with probabilities! Thus, this support does not have a Nash equilibrium.
\end{itemize}

We are left with having to consider one last support: $\{r_1,r_2,r_3\}$. It should be apparent that this case will simplify to one of the previous cases. Thus, we have found all the Nash equilibria:
$$(r_1,s_1),\;\left(\left\{{4\over5},{1\over5},0\right\},\left\{{2\over3},{1\over3}\right\}\right)\text{ and }\left(\left\{0,{1\over3},{2\over3}\right\},\left\{{1\over3},{2\over3}\right\}\right)$$


\item
\begin{enumerate}
\item Assuming ``walking in to each other'' gives both players a utility of $-1$ and ``avoiding each other'' a utility of 1, the bi matrix representation of this game is:

    $$\begin{tabular}{|c|c|c|}
\hline
 &$L$      &$R$     \\\hline
$L$&$(1,1)$&$(-1,-1)$\\\hline
$R$&$(-1,-1)$&$(1,1)$\\\hline
\end{tabular}$$
where $L,;R$ represent the step left and right strategies respectively.

\item Using best responses we have:
   $$\begin{tabular}{|c|c|c|}
\hline
 &$L$      &$R$     \\\hline
$L$&$(\underline{1},\underline{1})$&$(-1,-1)$\\\hline
$R$&$(-1,-1)$&$(\underline{1},\underline{1})$\\\hline
\end{tabular}$$
thus the two pure Nash equilibria are $\{L,L\}$ and $\{R,R\}$.


\item Assume player 1, plays the mixed strategy $\rho=(p,1-p)$ and player 2 plays the mixed strategy $\sigma=(q,1-q)$. By the equality of payoffs theorem we have:
  $$\begin{array}{@{}r@{\;}c@{\;}l@{\;\;\;\;\;\;\;\;\;\;}c@{\;\;\;\;\;\;\;\;\;\;}@{}r@{\;}c@{\;}l@{}}
u_1(L,\sigma)&=&u_1(R,\sigma)&\text{ and }&u_2(\rho,L)&=&u_2(\rho,R)\\[2mm]
q+(1-q)(-1)&=&q(-1)+(1-q)&\text{ and }&p+(1-p)(-1)&=&p(-1)+1-p\\[2mm]
q&=&{1\over 2}&\text{ and }&p&=&{1\over 2}\\[2mm]
\end{array}$$
thus
$p=q={1\over 2}$
The mixed Nash equilibria is $\left\{\left({1\over2},{1\over2}\right),\left({1\over2},{1\over2}\right)\right\}$

\end{enumerate}
\end{enumerate}
\end{document}
