\documentclass[12pt]{article}

%packages
\usepackage{graphicx}
\usepackage{amsmath}
\usepackage{mathdots}
\usepackage{amsthm}
\usepackage{amssymb}
\usepackage[dvips]{color}
\usepackage{fancyhdr}
\usepackage{pstricks}
\usepackage{pst-node}
\pagenumbering{arabic}
\usepackage{hyperref}
\usepackage{lscape}
%Margins etc...
\setlength{\textheight}{240mm}
\setlength{\topmargin}{-17mm} \setlength{\oddsidemargin}{-4mm}
\setlength{\textwidth}{166mm} \setlength{\parindent}{0mm}
\setlength{\marginparsep}{9mm} \setlength{\parskip}{3mm}

\begin{document}
\begin{center}
\Huge{Exploring Mixed Strategies with Sage\\\tiny{This sheet was last updated \today.} }
\end{center}

The following instructions are demonstrated in this video: \url{https://www.youtube.com/watch?v=Jfz9pU_2oHM&feature=youtu.be}:

\begin{enumerate}
    \item Create an account at \url{https://cloud.sagemath.com/}.
    \item Navigate to the Sage code snippet at: \url{https://gist.github.com/drvinceknight/7a72d952b91041905d8a}.
    \item Copy the code in to a Sage worksheet on cloud.sagemath.
\end{enumerate}

Using your notes try and investigate the following questions. Attempt to do this with the Sage interact but also using the mathematical techniques seen in the lecture.


\begin{enumerate}
\item Modify the input to consider the following game:
\begin{center}
\begin{tabular}{|c|c|}
\hline
$7,3$&$4,1$\\\hline
$1,2$&$6,4$\\
\hline
\end{tabular}
\end{center}
\item If player 1 is using the mixed strategy: $\rho=(.3,.7)$, what is player 2's best response?
\item What are the equilibria strategies?
\item What are the expected utilities to both players at each equilibria?
\end{enumerate}

\end{document}
