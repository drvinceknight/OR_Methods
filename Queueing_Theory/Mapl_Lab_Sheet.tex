\documentclass[12pt]{article}

%packages
\usepackage{graphicx}
\usepackage{amsmath}
\usepackage{mathdots}
\usepackage{amsthm}
\usepackage{amssymb}
\usepackage[dvips]{color}
\usepackage{fancyhdr}
\usepackage{pstricks}
\usepackage{pst-node}
\pagenumbering{arabic}
\usepackage{hyperref}
\usepackage{lscape}
%Margins etc...
\setlength{\textheight}{240mm}
\setlength{\topmargin}{-17mm} \setlength{\oddsidemargin}{-4mm}
\setlength{\textwidth}{166mm} \setlength{\parindent}{0mm}
\setlength{\marginparsep}{9mm} \setlength{\parskip}{3mm}

\begin{document}
\begin{center}
\Huge{Maple Lab Sheet}
\end{center}

In this final lab session, we will explore the maths package MAPLE.\\

\begin{enumerate}
\item Use MAPLE to perform the following calculations:
    \begin{itemize}
        \item $5+2$
        \item ${d\over dx}{x^2+5x+7}$
        \item Integrate the answer to the previous expression
        \item $\int_{0}^2{x^2+5x+7}dx$
        \item $\begin{pmatrix}2&0&4\\5&5&\pi\end{pmatrix}\begin{pmatrix}5&4\\e&5\\\ln2&1\over5\end{pmatrix}$
        \item $\sum_{i=0}^{n}p^i$
        \item Evaluate the previous expression for $p=.5$
        \item Evaluate the previous expression for $n=20$
    \end{itemize}
\item Create a plot of the following function: $f:x\to5x^2+2x+7$.
\item Create a plot of the following function: $f:(x,y)\to 5x^2y^3+2xy+7+20y$
\item Finally use MAPLE to \emph{verify/experiment} with results from the course notes:
    \begin{itemize}
        \item The expected value of the negative exponential distribution.
        \item The relationship between $r(t),R(t),F(t)$ and $f(t)$ for the Weibull distribution.
        \item Differentiate some of the cost functions from Inventory Theory to verify results from the notes.
        \item Anything else you can think of?
    \end{itemize}
\end{enumerate}

\end{document}